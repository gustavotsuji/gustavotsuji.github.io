
% ----------------------------------------------------------
% PARTE
% ----------------------------------------------------------
%\part{Preparação da pesquisa}

\section{Objetivo}

Este trabalho se baseia em um experimento quantitativo sobre problemas relacionados a segmentação e classificação de dados de uma base de dados grande, utilizando algoritmos de aprendizagem supervisionada e não supervisionada. Para resolver o problema de segmentação, este trabalho irá abordar o algoritmo de K Médias (não supervisionado) e para os casos de classificação, Regressão Logística e \emph{Random Forest} (supervisionados). Também serão feitas interpretações, análises e comparações, levantando aspectos positivos e negativos de cada metodologia.

\section{Cronograma}

\subsection{Primeiro semestre}

Para este período, ficou decidido estudar o contexto atual de como as empresas estão lidando com os horizontes abertos pelo \emph{Big Data} e seus desafios. Também foi definido que é necessário estudar sobre princípios básicos de \emph{Machine Learning} para a resolução de problemas de clusterização e classificação, aprofundando sobre detalhes dos conceitos por trás dos algoritmos e de sua implementação.

\begin{ganttchart}[
    canvas/.append style={fill=none, draw=black!5, line width=.75pt},
    hgrid style/.style={draw=black!5, line width=.75pt},
    vgrid={*1{draw=black!5, line width=.75pt}},
    % today=19,
    today rule/.style={
      draw=black!64,
      dash pattern=on 3.5pt off 4.5pt,
      line width=1.5pt
    },
    today label = HOJE,
    today label font=\small\bfseries,
    title/.style={draw=none, fill=none},
    title label font=\bfseries\footnotesize,
    title label node/.append style={below=7pt},
    include title in canvas=false,
    bar label font=\mdseries\small\color{black!70},
    bar label node/.append style={left=2cm},
    bar/.append style={draw=none, fill=black!63},
    bar incomplete/.append style={fill=barblue},
    bar progress label font=\mdseries\footnotesize\color{black!70},
    group incomplete/.append style={fill=groupblue},
    group left shift=0,
    group right shift=0,
    group height=.5,
    group peaks tip position=0,
    group label node/.append style={left=.6cm},
    group progress label font=\bfseries\small,
    link/.style={-latex, line width=1.5pt, linkred},
    link label font=\scriptsize\bfseries,
    link label node/.append style={below left=-2pt and 0pt}
  ]{1}{18}
  \gantttitle[
    title label node/.append style={below left=7pt and -3pt}
  ]{SEMANAS:\quad1}{1}
  \gantttitlelist{2,...,18}{1} \\
  \ganttgroup[]{Refer\^encia Te\'orica}{1}{18} \\
  \ganttbar[
    name=WBS1A
  ]{Big Data}{1}{3} \\
  \ganttbar[
    name=WBS1A
  ]{Machine Learning}{2}{4} \\
  \ganttbar[
    name=WBS1A
  ]{Regress\~ao Log\'istica}{4}{7} \\
  \ganttbar[
     name=WBS1B
  ]{Random Forest}{7}{11} \\
  \ganttbar[
    name=WBS1C
  ]{K M\'edias}{11}{15} \\[grid]
  \ganttgroup[]{Monografia}{15}{18} \\
  \ganttbar[
    name=WBS1D
  ]{Elaboração do modelo}{15}{16}\\
  \ganttbar[
    name=WBS1D
  ]{Ajustes no texto}{16}{18}
  % \ganttgroup[]{Desenvolvimento}{20}{40} \\
  % \ganttbar[]{Estudos com Spark}{20}{28} \\
  % \ganttbar[]{Comparac\~ao entre modelos}{26}{31} \\
  % \ganttbar[]{}{26}{31} \\
  % \ganttbar[]{\textbf{WBS 2.3} Activity G}{9}{10}
  \end{ganttchart}

\subsection{Segundo semestre}

Para este período, será feito uma revisão sobre o conteúdo apresentado na primeira versão da monografia, ajustes em relação a proposta do trabalho e aplicação dos algoritmos. A infraestrutura e tecnologia a serem utlizadas serão pensados e revistos durante o segundo semestre, visto que a execução sobre uma grande base de dados é complexa e custosa. Também iremos escolher uma base grande apropriada para a execução dos algoritmos.

\begin{ganttchart}[
    canvas/.append style={fill=none, draw=black!5, line width=.75pt},
    hgrid style/.style={draw=black!5, line width=.75pt},
    vgrid={*1{draw=black!5, line width=.75pt}},
    % today=19,
    today rule/.style={
      draw=black!64,
      dash pattern=on 3.5pt off 4.5pt,
      line width=1.5pt
    },
    today label = HOJE,
    today label font=\small\bfseries,
    title/.style={draw=none, fill=none},
    title label font=\bfseries\footnotesize,
    title label node/.append style={below=7pt},
    include title in canvas=false,
    bar label font=\mdseries\small\color{black!70},
    bar label node/.append style={left=2cm},
    bar/.append style={draw=none, fill=black!63},
    bar incomplete/.append style={fill=barblue},
    bar progress label font=\mdseries\footnotesize\color{black!70},
    group incomplete/.append style={fill=groupblue},
    group left shift=0,
    group right shift=0,
    group height=.5,
    group peaks tip position=0,
    group label node/.append style={left=.6cm},
    group progress label font=\bfseries\small,
    link/.style={-latex, line width=1.5pt, linkred},
    link label font=\scriptsize\bfseries,
    link label node/.append style={below left=-2pt and 0pt}
  ]{1}{18}
  \gantttitle[
    title label node/.append style={below left=7pt and -3pt}
  ]{SEMANAS:\quad1}{1}
  \gantttitlelist{2,...,18}{1} \\
  \ganttgroup[]{Algoritmos}{1}{10} \\
  \ganttbar[
    name=WBS1A
  ]{Escrever código}{1}{2} \\
  \ganttbar[
    name=WBS1A
  ]{Estudar Spark}{1}{4} \\
  \ganttbar[
    name=WBS1A
  ]{Testes}{4}{7} \\
  \ganttbar[
    name=WBS1A
  ]{Execução}{7}{10} \\[grid]
  \ganttgroup[]{Infraestrutura}{7}{10} \\
  \ganttbar[
    name=WBS1A
  ]{Estudar alternativas}{7}{8} \\
  \ganttbar[
    name=WBS1A
  ]{M\'aquinas remotas}{8}{9} \\
  \ganttbar[
    name=WBS1A
  ]{Criar um protótipo}{9}{10}\\[grid]
  \ganttgroup[]{Monografia}{9}{18} \\
  \ganttbar[
    name=WBS1D
  ]{Revisão do orientador}{9}{10}\\
  \ganttbar[
    name=WBS1D
  ]{Resultados}{9}{12}
  \\
  \ganttbar[
    name=WBS1D
  ]{Conclusão}{12}{15}
  \\
  \ganttbar[
    name=WBS1D
  ]{Ajuste final}{15}{18}
  % \ganttgroup[]{Desenvolvimento}{20}{40} \\
  % \ganttbar[]{Estudos com Spark}{20}{28} \\
  % \ganttbar[]{Comparac\~ao entre modelos}{26}{31} \\
  % \ganttbar[]{}{26}{31} \\
  % \ganttbar[]{\textbf{WBS 2.3} Activity G}{9}{10}
  \end{ganttchart}

% \begin{landscape}
% \centering
% \begin{vplace}[0.7]



% \end{vplace}
% \end{landscape}

% ----------------------------------------------------------
