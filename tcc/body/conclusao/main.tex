% ---
% Conclusão
% ---
\chapter{Considerações Finais}
% ---

\section{Segmentação}
O K Médias é um algoritmo que depende de uma quantidade inicial de \emph{clusters}. Sem uma análise dos dados ou alguma proposta e/ou planejamento inicial, a única alternativa é definir algum valor aleatório e fazer uma análise sobre os \emph{clusters} gerados. Quando se trabalha com poucos dados, seria possível calcular o \emph{silhoutte score} para ter um embasamento sobre a quantidade de \emph{cluster} inicial. Contudo, ao se trabalhar com um grande volume, uma possível alternativa poderia ser retirar amostras para calcular um valor médio. Neste trabalho, apesar da escolha inicial não ser o foco do estudo, utilizou-se uma análise superficial a partir de uma pequena amostra na qual gerou-se \emph{boxplots} para visualizar as características dos dados em seus respectivos segmentos. Já na execução sobre a base completa, a verificação foi feita com a descrição de algumas métricas (média, desvio padrão, valor mínimo e máximo). Como o resultado da clusterização do K Médias serviu de entrada de informação para os algoritmos de classificação, não foi considerado necessário uma verificação mais profunda sobre os \emph{clusters}. Se houvesse algum resultado discrepante, isso se verificaria na classificação dos dados.

\section{Classificação}

Ambos os algoritmos conseguiram classificar os registros com sucesso. Dado a segmentação gerada pela K Médias, tanto a Regressão Logística como a \emph{Random Forest} obtiveram bons índices.

A Regressão Logística, quando executado em bases com dados que possuem um comportamento mais linear, tende a ser mais estável. Quando a base não possui tal característica, um dos problemas que pode ocorrer é o \emph{overfitting}. Há alguma saídas para se contornar isso, que é a adicionando parâmetros l2 ou l1 na execução da Regressão Logística. Na base da \emph{Loan Club}, verificou-se que isto não representou um problema muito significativo mas no caso de tentar melhorar os resultados, seria interessante um estudo em cada parâmetro disponível na execução do algoritmo. Há dois pontos de atenção para a regressão logística: caso seja intrínsico trabalhar com as variáveis categóricas, é necessário a transformação dessas \emph{features} em variáveis \emph{dummies}. Além disso, a normalização dos dados é recomendável para que não haja uma distorção muito grande nos cálculos do K Médias.

O \emph{Random Forest} possui vantagens como a possibilidade de trabalhar com base de dados que não tenham um comportamento linear. Além disso, elas podem tratar variáveis categóricas, fato que é complicado quando se trabalha com regressão logística, já que a \emph{Random Forest} são árvores de decisões. Com o uso de \emph{bagging} ou \emph{boosting}, é possível trabalhar com uma grande quantidade de variáveis, além de realizar diversos treinos. Além disso, a \emph{Random Forest} pode selecionar quais são as variáveis mais relevantes independente da quantidade de váriaveis que ela receba inicialmente.

Por fim, é importante ressaltar que os resultados provenientes da execução dos algoritmos não representam uma verdade absoluta e inquestionável. Trata-se de uma modelagem que busca se aproximar da realidade mas que inerentemente fica condicionada a limitações. Devem ser compreendidos como um possível direcionamento ou solução que possui um embasamento científico. Todavia, os algoritmos de \emph{Machine Learning} vem se mostrando ferramentas poderosas e que podem ser úteis em diversas áreas de setores distintos.

\section{Sugestão para trabalhos futuros}
Este trabalhou limitou-se a estudar apenas 1 algoritmo de segmentação e 2 de classificação. Existem outros modelos que se adequam melhor a cada situação. Seria interessante que fossem estudados para ter um parâmetro de comparação.
Além disso, para cada algoritmo também existem ajustes que podem ser feitos para se obter um resultado ainda melhor.
Não obstante a esses pontos, também seria possível explorar ainda mais o banco de dados com os dados que foram excluídos propositalmente para reduzir o escopo do trabalho.


%\lipsum[31-33]

