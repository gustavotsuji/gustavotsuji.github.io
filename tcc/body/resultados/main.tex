% ----------------------------------------------------------
% PARTE
% ----------------------------------------------------------
\part{Resultados}
% ----------------------------------------------------------

\section{Kaggle}

In 2010, Kaggle was founded as a platform for predictive modelling and analytics competitions on which companies and researchers post their data and statisticians and data miners from all over the world compete to produce the best models. This crowdsourcing approach relies on the fact that there are countless strategies that can be applied to any predictive modelling task and it is impossible to know at the outset which technique or analyst will be most effective. Kaggle also hosts recruiting competitions in which data scientists compete for a chance to interview at leading data science companies like Facebook, Winton Capital, and Walmart.

In April 2015, Kaggle released the first version of their Scripts product onto their platform. Scripts allows users to write, run, and publicly share their code on Kaggle. In January 2016, Kaggle released their Datasets product, making a selection of public datasets available on Kaggle. Each datasets has Scripts enabled, as well as a dedicated forum, allowing for conversation and collaboration between data scientists and the work they are doing on each dataset. On 8 July 2016, Kaggle renamed its Scripts product to Kernels

\subsection{Loan Club}

We are the world’s largest online credit marketplace, facilitating personal loans, business loans, and financing for elective medical procedures. Borrowers access lower interest rate loans through a fast and easy online or mobile interface. Investors provide the capital to enable many of the loans in exchange for earning interest. We operate fully online with no branch infrastructure, and use technology to lower cost and deliver an amazing experience. We pass the cost savings to borrowers in the form of lower rates and investors in the form of attractive returns. We’re transforming the banking system into a frictionless, transparent and highly efficient online marketplace, helping people achieve their financial goals everyday.


\section{Tecnologias}

\subsection{Scikit Learn}

If you are a Python programmer or you are looking for a robust library you can use to bring machine learning into a production system then a library that you will want to seriously consider is scikit-learn.

In this post you will get an overview of the scikit-learn library and useful references of where you can learn more.

Scikit-learn was initially developed by David Cournapeau as a Google summer of code project in 2007.

Later Matthieu Brucher joined the project and started to use it as apart of his thesis work. In 2010 INRIA got involved and the first public release (v0.1 beta) was published in late January 2010.

The project now has more than 30 active contributors and has had paid sponsorship from INRIA, Google, Tinyclues and the Python Software Foundation.

Scikit-learn provides a range of supervised and unsupervised learning algorithms via a consistent interface in Python.

It is licensed under a permissive simplified BSD license and is distributed under many Linux distributions, encouraging academic and commercial use.

The library is built upon the SciPy (Scientific Python) that must be installed before you can use scikit-learn. This stack that includes:

NumPy: Base n-dimensional array package
SciPy: Fundamental library for scientific computing
Matplotlib: Comprehensive 2D/3D plotting
IPython: Enhanced interactive console
Sympy: Symbolic mathematics
Pandas: Data structures and analysis
Extensions or modules for SciPy care conventionally named SciKits. As such, the module provides learning algorithms and is named scikit-learn.

The vision for the library is a level of robustness and support required for use in production systems. This means a deep focus on concerns such as easy of use, code quality, collaboration, documentation and performance.

\subsection{Spark}

Apache Spark provides programmers with an application programming interface centered on a data structure called the resilient distributed dataset (RDD), a read-only multiset of data items distributed over a cluster of machines, that is maintained in a fault-tolerant way.[2] It was developed in response to limitations in the MapReduce cluster computing paradigm, which forces a particular linear dataflow structure on distributed programs: MapReduce programs read input data from disk, map a function across the data, reduce the results of the map, and store reduction results on disk. Spark's RDDs function as a working set for distributed programs that offers a (deliberately) restricted form of distributed shared memory.[3]

The availability of RDDs facilitates the implementation of both iterative algorithms, that visit their dataset multiple times in a loop, and interactive/exploratory data analysis, i.e., the repeated database-style querying of data. The latency of such applications (compared to Apache Hadoop, a popular MapReduce implementation) may be reduced by several orders of magnitude.[2][4] Among the class of iterative algorithms are the training algorithms for machine learning systems, which formed the initial impetus for developing Apache Spark.[5]

Spark MLlib is a distributed machine learning framework on top of Spark Core that, due in large part to the distributed memory-based Spark architecture, is as much as nine times as fast as the disk-based implementation used by Apache Mahout (according to benchmarks done by the MLlib developers against the Alternating Least Squares (ALS) implementations, and before Mahout itself gained a Spark interface), and scales better than Vowpal Wabbit.[14] Many common machine learning and statistical algorithms have been implemented and are shipped with MLlib which simplifies large scale machine learning pipelines, including:

summary statistics, correlations, stratified sampling, hypothesis testing, random data generation[15]
classification and regression: support vector machines, logistic regression, linear regression, decision trees, naive Bayes classification
collaborative filtering techniques including alternating least squares (ALS)
cluster analysis methods including k-means, and Latent Dirichlet Allocation (LDA)
dimensionality reduction techniques such as singular value decomposition (SVD), and principal component analysis (PCA)
feature extraction and transformation functions
optimization algorithms such as stochastic gradient descent, limited-memory BFGS (L-BFGS)

\section{K Médias}

\section{Regressão Logística]}

\section{Random Forest}








 \label{tab:daypack}
    \begin{tabularx}{\textwidth}{p{.3\textwidth}X}
    \caption{Tabela de campos disponíveis em Loan Club}\\
    \toprule
    \textbf{Coluna} & \textbf{Descrição} \\[6pt]
    \midrule
    \endhead

addr\textunderscore state & The state provided by the borrower in the loan application\\
annual\textunderscore inc & The self-reported annual income provided by the borrower during registration.\\
annual\textunderscore inc\textunderscore joint & The combined self-reported annual income provided by the co-borrowers during registration\\
application\textunderscore type & Indicates whether the loan is an individual application or a joint application with two co-borrowers\\
collection\textunderscore recovery\textunderscore fee & post charge off collection fee\\
collections\textunderscore 12\textunderscore mths\textunderscore ex\textunderscore med & Number of collections in 12 months excluding medical collections\\
delinq\textunderscore 2yrs & The number of 30+ days past-due incidences of delinquency in the borrower's credit file for the past 2 years\\
desc & Loan description provided by the borrower\\
dti & A ratio calculated using the borrower’s total monthly debt payments on the total debt obligations, excluding mortgage and the requested LC loan, divided by the borrower’s self-reported monthly income.\\
dti\textunderscore joint & A ratio calculated using the co-borrowers' total monthly payments on the total debt obligations, excluding mortgages and the requested LC loan, divided by the co-borrowers' combined self-reported monthly income\\
earliest\textunderscore cr\textunderscore line & The month the borrower's earliest reported credit line was opened\\
emp\textunderscore length & Employment length in years. Possible values are between 0 and 10 where 0 means less than one year and 10 means ten or more years. \\
emp\textunderscore title & The job title supplied by the Borrower when applying for the loan.*\\
fico\textunderscore range\textunderscore high & The upper boundary range the borrower’s FICO at loan origination belongs to.\\
fico\textunderscore range\textunderscore low & The lower boundary range the borrower’s FICO at loan origination belongs to.\\
funded\textunderscore amnt & The total amount committed to that loan at that point in time.\\
funded\textunderscore amnt\textunderscore inv & The total amount committed by investors for that loan at that point in time.\\
grade & LC assigned loan grade\\
home\textunderscore ownership & The home ownership status provided by the borrower during registration. Our values are: RENT, OWN, MORTGAGE, OTHER.\\
id & A unique LC assigned ID for the loan listing.\\
initial\textunderscore list\textunderscore status & The initial listing status of the loan. Possible values are – W, F\\
inq\textunderscore last\textunderscore 6mths & The number of inquiries in past 6 months (excluding auto and mortgage inquiries)\\
installment & The monthly payment owed by the borrower if the loan originates.\\
int\textunderscore rate & Interest Rate on the loan\\
is\textunderscore inc\textunderscore v & Indicates if income was verified by LC, not verified, or if the income source was verified\\
issue\textunderscore d & The month which the loan was funded\\
last\textunderscore credit\textunderscore pull\textunderscore d & The most recent month LC pulled credit for this loan\\
last\textunderscore fico\textunderscore range\textunderscore high & The upper boundary range the borrower’s last FICO pulled belongs to.\\
last\textunderscore fico\textunderscore range\textunderscore low & The lower boundary range the borrower’s last FICO pulled belongs to.\\
last\textunderscore pymnt\textunderscore amnt & Last total payment amount received\\
last\textunderscore pymnt\textunderscore d & Last month payment was received\\
loan\textunderscore amnt & The listed amount of the loan applied for by the borrower. If at some point in time, the credit department reduces the loan amount, then it will be reflected in this value.\\
loan\textunderscore status & Current status of the loan\\
member\textunderscore id & A unique LC assigned Id for the borrower member.\\
mths\textunderscore since\textunderscore last\textunderscore delinq & The number of months since the borrower's last delinquency.\\
mths\textunderscore since\textunderscore last\textunderscore major\textunderscore derog & Months since most recent 90-day or worse rating\\
mths\textunderscore since\textunderscore last\textunderscore record & The number of months since the last public record.\\
next\textunderscore pymnt\textunderscore d & Next scheduled payment date\\
open\textunderscore acc & The number of open credit lines in the borrower's credit file.\\
out\textunderscore prncp & emaining outstanding principal for total amount funded\\
out\textunderscore prncp\textunderscore inv & Remaining outstanding principal for portion of total amount funded by investors\\
policy\textunderscore code & "publicly available policy\textunderscore code=1\\
new products not publicly available policy\textunderscore code=2"\\
pub\textunderscore rec & Number of derogatory public records\\
purpose & A category provided by the borrower for the loan request. \\
pymnt\textunderscore plan & Indicates if a payment plan has been put in place for the loan\\
recoveries & post charge off gross recovery\\
revol\textunderscore bal & Total credit revolving balance\\
revol\textunderscore util & Revolving line utilization rate, or the amount of credit the borrower is using relative to all available revolving credit.\\
sub\textunderscore grade & LC assigned loan subgrade\\
term & The number of payments on the loan. Values are in months and can be either 36 or 60.\\
title & The loan title provided by the borrower\\
total\textunderscore acc & The total number of credit lines currently in the borrower's credit file\\
total\textunderscore pymnt & Payments received to date for total amount funded\\
total\textunderscore pymnt\textunderscore inv & Payments received to date for portion of total amount funded by investors\\
total\textunderscore rec\textunderscore int & Interest received to date\\
total\textunderscore rec\textunderscore late\textunderscore fee & Late fees received to date\\
total\textunderscore rec\textunderscore prncp & Principal received to date\\
verified\textunderscore status\textunderscore joint & Indicates if the co-borrowers' joint income was verified by LC, not verified, or if the income source was verified\\
zip\textunderscore code & The first 3 numbers of the zip code provided by the borrower in the loan application.\\
open\textunderscore acc\textunderscore 6m & Number of open trades in last 6 months\\
open\textunderscore il\textunderscore 6m & Number of currently active installment trades\\
open\textunderscore il\textunderscore 12m & Number of installment accounts opened in past 12 months\\
open\textunderscore il\textunderscore 24m & Number of installment accounts opened in past 24 months\\
mths\textunderscore since\textunderscore rcnt\textunderscore il & Months since most recent installment accounts opened\\
total\textunderscore bal\textunderscore il & Total current balance of all installment accounts\\
il\textunderscore util & Ratio of total current balance to high credit/credit limit on all install acct\\
open\textunderscore rv\textunderscore 12m & Number of revolving trades opened in past 12 months\\
open\textunderscore rv\textunderscore 24m & Number of revolving trades opened in past 24 months\\
max\textunderscore bal\textunderscore bc & Maximum current balance owed on all revolving accounts\\
all\textunderscore util & Balance to credit limit on all trades\\
total\textunderscore rev\textunderscore hi\textunderscore lim & Total revolving high credit/credit limit\\
inq\textunderscore fi & Number of personal finance inquiries\\
total\textunderscore cu\textunderscore tl & Number of finance trades\\
inq\textunderscore last\textunderscore 12m & Number of credit inquiries in past 12 months\\
acc\textunderscore now\textunderscore delinq & The number of accounts on which the borrower is now delinquent.\\
tot\textunderscore coll\textunderscore amt & Total collection amounts ever owed \\
tot\textunderscore cur\textunderscore bal & Total current balance of all accounts \\

\bottomrule

\end{tabularx}
% ---
% primeiro capitulo de Resultados
% ---
%\chapter{Lectus lobortis condimentum}
% ---

% ---
%\section{Vestibulum ante ipsum primis in faucibus orci luctus et ultrices
%posuere cubilia Curae}
% ---

%\lipsum[21-22]

% ---
% segundo capitulo de Resultados
% ---
%\chapter{Nam sed tellus sit amet lectus urna ullamcorper tristique interdum
%elementum}
% ---

% ---
%\section{Pellentesque sit amet pede ac sem eleifend consectetuer}
% ---

%\lipsum[24]

% ----------------------------------------------------------
% Finaliza a parte no bookmark do PDF
% para que se inicie o bookmark na raiz
% e adiciona espaço de parte no Sumário
% ----------------------------------------------------------
%\phantompart