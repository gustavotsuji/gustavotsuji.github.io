% ----------------------------------------------------------
% Introdução (exemplo de capítulo sem numeração, mas presente no Sumário)
% ----------------------------------------------------------
\chapter*[Introdução]{Introdução}
\addcontentsline{toc}{chapter}{Introdução}
% ----------------------------------------------------------

Na \emph{internet}, é possível encontrar mais de 60 trilhões de páginas indexadas pelo Google \cite{GOO01}. Só o Facebook possui \emph{data warehouses}\footnotemark \footnotetext{\emph{data warehouses} são banco de dados otimizados para consultas e análise de dados e geração de relatório em detrimento de operações transacionais} com mais de 300 \emph{petabytes}, tendo um tráfego de mais de 600 \emph{terabytes} diários \cite{FAC01}. Como é possível notar, a tecnologia evoluiu a ponto de tornar o armazenamento de volume de dados um grande desafio. Mas, ao mesmo tempo, abriu portas para novas possibilidades.

O conteúdo dessas páginas ou operações executadas freneticamente todos os dias geram dados que, segundo \citeonline{BAEZA}, são objetos brutos que trazem pouco ou nenhum significado, armazenados de forma estruturada ou não. Por trás desses dados existe o que o autor chama de informação. A informação, então, seria uma interpretação do dado dentro de um contexto com um ganho cognitivo. A utilização dela produz o conhecimento e a aprendizagem, o que permite o desenvolvimento de novos produtos ou serviços como um melhor entendimento sobre comportamento dos clientes e do funcionamento de processos, otimização da produção, entre outras melhorias.

Por conta da dificuldade computacional em não só armazenar como também analisar e monitorar esse volume de dados que nasceu o \emph{Big Data}. A interpretação e extração de informações possibilitam uma melhor compreensão de vários aspectos, tanto nas esferas micro e macro na empresa. A utilização de \emph{Big Data} a favor da companhia pode conferir uma grande vantagem competitiva, bem como uma diferenciação, sendo considerado um ativo estratégico muito valioso.

Este trabalho visa, portanto, estudar conceitos teóricos estatísticos que analisam os dados, os algoritmos que criam as informações, bem como tecnologias que auxiliam o processamento e execução do algoritmo em larga escala.
