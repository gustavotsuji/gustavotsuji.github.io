% ----------------------------------------------------------
% Introdução (exemplo de capítulo sem numeração, mas presente no Sumário)
% ----------------------------------------------------------
\chapter*[Introdução]{Introdução}
\addcontentsline{toc}{chapter}{Introdução}
% ----------------------------------------------------------

A tecnologia evoluiu a ponto de tornar possível o armazenamento de volume de dados um grande desafio. Na internet, é possível encontrar mais de 60 trilhões de páginas indexadas pelo Google \cite{GOO01}. Só o Facebook possui warehouse com mais de 300 petabytes, tendo um tráfego de mais de 600 terabytes diários \cite{FAC01}.

Por trás desses dados brutos armazenados de forma estruturada ou não, existe o que \citeonline{BAEZA} chama de informação. Em geral, os dados são objetos brutos que trazem pouco ou nenhum significado. A informação, então, refere-se a uma interpretação do dado dentro de um contexto com um ganho cognitivo. A utilização dessa informação para qualquer fim produz o conhecimento. 

Por conta da dificuldade computacional em não só armazenar como analisar e monitorar esse volume de dados que nasceu o Big Data. A análise e extração de informações possibilitam uma melhor compreensão de vários aspectos, micro e macro na empresa. A utilização de Big Data a favor da companhia pode conferir uma grande vantagem competitiva, bem como uma diferenciação, sendo considerado um ativo estratégico muito valioso.

Este trabalho visa estudar conceitos teóricos estatísticos que analisam os dados, os algoritmos que criam as informações, bem como tecnologias que auxiliam o processo como um todo.
